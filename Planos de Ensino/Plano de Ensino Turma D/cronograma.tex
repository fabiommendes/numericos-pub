\section{Cronograma}
\begin{tabularx}{0.9\textwidth}{cccX}
\toprule
\textbf{Semana} & \textbf{Aula} & \textbf{Data} & \textbf{Conteúdo}\\
\toprule
\multirow{2}*{\textbf{01}}
& \cellcolor[gray]{.9}1 & \cellcolor[gray]{.9}06/03 & \cellcolor[gray]{.9}\textit{Apresentação do curso} \\
& 2 & 08/03 & \textit{Ferramentas} \\
\midrule
\multirow{2}*{\textbf{02}}
& \cellcolor[gray]{.9}3 & \cellcolor[gray]{.9}13/03 & \cellcolor[gray]{.9}\textit{Introdução ao Python} \\
& 4 & 15/03 & \textit{Introdução ao Python científico} \\
\midrule
\multirow{2}*{\textbf{03}}
& \cellcolor[gray]{.9}5 & \cellcolor[gray]{.9}20/03 & \cellcolor[gray]{.9}\textit{Zeros de Funções} \\
& 6 & 22/03 & \textit{Zeros de Funções} \\
\midrule
\multirow{2}*{\textbf{04}}
& \cellcolor[gray]{.9}7 & \cellcolor[gray]{.9}27/03 & \cellcolor[gray]{.9}\textit{Mínimo de funções} \\
& 8 & 29/03 & \textit{Mínimo de funções de várias variáveis} \\
\midrule
\multirow{2}*{\textbf{05}}
& \cellcolor[gray]{.9}9 & \cellcolor[gray]{.9}03/04 & \cellcolor[gray]{.9}\textit{Interpolação} \\
& 10 & 05/04 & \textit{Splines} \\
\midrule
\multirow{2}*{\textbf{06}}
& \cellcolor[gray]{.9}11 & \cellcolor[gray]{.9}10/04 & \cellcolor[gray]{.9}\textit{Ajuste de curvas: retas e mínimos quadrados} \\
& 12 & 12/04 & \textit{Mínimos quadrados em modelos arbitrários} \\
\midrule
\multirow{2}*{\textbf{07}}
& \cellcolor[gray]{.9}13 & \cellcolor[gray]{.9}17/04 & \cellcolor[gray]{.9}\textit{Revisão} \\
& - & 19/04 & \textbf{Prova 01} \\
\midrule
\multirow{2}*{\textbf{08}}
& \cellcolor[gray]{.9}14 & \cellcolor[gray]{.9}24/04 & \cellcolor[gray]{.9}\textit{Resolução da Prova 01} \\
& 15 & 26/04 & \textit{Resolução de sistemas lineares} \\
\midrule
\multirow{2}*{\textbf{09}}
& \cellcolor[gray]{.9}- & \cellcolor[gray]{.9}01/05 & \cellcolor[gray]{.9}\textbf{Feriado} \\
& 16 & 03/05 & \textit{Formas matriciais especiais} \\
\midrule
\multirow{2}*{\textbf{10}}
& \cellcolor[gray]{.9}17 & \cellcolor[gray]{.9}08/05 & \cellcolor[gray]{.9}\textit{Métodos iterativos de fatoração} \\
& 18 & 10/05 & \textit{Comparação de desempenho} \\
\midrule
\multirow{2}*{\textbf{11}}
& \cellcolor[gray]{.9}19 & \cellcolor[gray]{.9}15/05 & \cellcolor[gray]{.9}\textit{Integração numérica} \\
& 20 & 17/05 & \textit{Regras avançadas de integração numérica} \\
\midrule
\multirow{2}*{\textbf{12}}
& \cellcolor[gray]{.9}21 & \cellcolor[gray]{.9}22/05 & \cellcolor[gray]{.9}\textit{Erro de regras de quadratura} \\
& 22 & 24/05 & \textit{Soluções numéricas de EDO} \\
\midrule
\multirow{2}*{\textbf{13}}
& \cellcolor[gray]{.9}23 & \cellcolor[gray]{.9}29/05 & \cellcolor[gray]{.9}\textit{Métodos Runge-Kutta} \\
& 24 & 31/05 & \textit{Sistemas de EDO} \\
\midrule
\multirow{2}*{\textbf{14}}
& \cellcolor[gray]{.9}25 & \cellcolor[gray]{.9}05/06 & \cellcolor[gray]{.9}\textit{Aplicações a sistemas físicos} \\
& 26 & 07/06 & \textit{Método das diferenças finitas} \\
\midrule
\multirow{2}*{\textbf{15}}
& \cellcolor[gray]{.9}27 & \cellcolor[gray]{.9}12/06 & \cellcolor[gray]{.9}\textit{Fontes de erros} \\
& 28 & 14/06 & \textit{Revisão} \\
\midrule
\multirow{2}*{\textbf{16}}
& \cellcolor[gray]{.9}- & \cellcolor[gray]{.9}19/06 & \cellcolor[gray]{.9}\textbf{Prova 02} \\
& 29 & 21/06 & \textit{Resolução da Prova 02} \\
\midrule
\multirow{2}*{\textbf{17}}
& \cellcolor[gray]{.9}- & \cellcolor[gray]{.9}26/06 & \cellcolor[gray]{.9}\textit{Revisão de Provas} \\
& - & 28/06 & \textbf{Prova Substitutiva} \\
\midrule
\multirow{2}*{\textbf{18}}
& \cellcolor[gray]{.9}- & \cellcolor[gray]{.9}03/07 & \cellcolor[gray]{.9}\textit{Menções Finais.  Revisão de notas} \\
& - & 05/07 & \textit{Submissão das menções finais no sistema acadêmico} \\
\bottomrule
\end{tabularx}
